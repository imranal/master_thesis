% !TEX spellcheck = en
\documentclass[main.tex]{subfiles} 

\begin{document}

\section{Conclusion}
\label{sec:6}

The software we have developed so far has room for much improvement. Firstly, the hyperstreamline
module does not handle closed-hyperstreamlines. This is an important case to handle,
as to prevent infinite cycling during hyperstreamline calculation \cite{WM06}.Topological
features such as degenerate points are also important cases that must be handeled. 
The topology of a tensor field is the topology of its eigenvectors. In the case of two or
more eigenvalues being equal, at least one eigenvector
is linearly dependent. As a consequence, hyperstreamlines at such points, can
branch out through multiple paths. Finally, perhaps the most important physical characteristic 
of any tensor field is its time dependency. Handling an unsteady tensor field is 
important as many physical processes are unsteady.
\\

One of the main challenges of tensor field visualization is the difficulty there lies in
adapting same techniques across datasets with different physical attributes. Examples of 
such fields include stress and strain tensors, rate of deformation tensor, and the diffusion 
tensor. The physical meaning of tensors can greatly impact how they should be visualized, 
even when the mathematical representations of these tensors are the same 
(as we have shown in \ref{sec:2}) \cite{HHK14}. Using differntial geometry, we have demonstrated 
a new method, where we solve the geodesic differential equations and apply similar techniques 
as hyperstreamlines. The method itself was applied on metric tensor. It can easily be 
extended to any other second order tensor. Though, there are limitation, such as 
the Christoffel symbol of second kind exists only if the metric is non-singular. We showed
that the momentum flux density is one such tensor which can not be applied. 

We can take the geodesics one step further, if we can manage to combine hyperstreamlines
techniques using geodesics to determine the pricipal direction. This hybrid geodesic-hyperstreamline
method requires further investigation.
\\

The important finding of this thesis are as following : There is a gaping disparity for
readily available free software which permit the user multiple visualization methods for
second order (or higher order) tensor fields. As such, we set ourselves upon the daunting
task of creating our own tensor module (even though we limited ourselves to a few methods).
However, the process it self was quite revelatory. To create a module from almost scratch
is an exiciting task, but still quite difficult. As such, much of the focus has been
in the implementation process itself.


\end{document}