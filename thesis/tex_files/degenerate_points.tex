% !TEX spellcheck = en
\documentclass[12pt,twoside,onecolumn]{article}
\usepackage{a4}
\usepackage[margin=0.95in]{geometry}
\usepackage[latin1]{inputenc}  % Bruk latin-1 tegnsett
\usepackage[UKenglish]{babel}  
\usepackage{hyperref}  % for making links click-able
\usepackage{amsmath} % math package
\usepackage{amsthm}
\theoremstyle{definition}
\newtheorem{thm}{Theorem}
\newtheorem{defn}{Definition}
\begin{document}

\section{Analytical Tensor Fields}
Given the symmetric second rank tensor
\begin{align*} 
T&=
\begin{bmatrix}
    T_{11} &  T_{12} \\
    T_{12}  &  T_{22}
\end{bmatrix},
\end{align*}
For this tensor field, degenerate points satisfy the condition
\begin{align} \label{degenerate_cond}
\left\{\begin{matrix}
 \frac{T_{11} - T_{22}}{2} &=0 \\
 T_{12} &=0
\end{matrix}\right.
\end{align}
Assume that the tensor components can be expressed in the vicinity of degenerate point 
$\vec{x}_0 = (x_0,y_0)$ as Taylor expansions that start with homogeneous polynomials 
for $m > 0$
\begin{align}
\left\{\begin{matrix}
 \frac{T_{11} - T_{22}}{2} &\approx P_m(x - x_0, y - y_0) + \cdots \\
 T_{12} &\approx Q_m(x - x_0, y - y_0) + \cdots
\end{matrix}\right.
\end{align}

\subsection{Analytical Examples, see pages 134 - 139}
\begin{align*}
T&=
\begin{bmatrix}
    \frac{x^2 + 4xy + y^2}{2} &  -x^2 + y^2 \\
    -x^2 + y^2  &  \frac{-x^2 - 4xy - y^2}{2}
\end{bmatrix},
\end{align*}
which has a multiple degenerate point (see definition : \ref{definition}) at the origin.
\\
\\
$\left<\mbox{ sepratrices of multiple degenerate points are described in page 131 - 133
}\right>$
\\
\\
More such examples are detailed in the same section, with description of the sepratrices
that occur.

\section{Algorithm to Extract the Topology of a Tensor Field}
In tensor fields different types of degenerate points can occur that correspond to different
local patterns of neighboring hyperstreamlines. These patterns are determined by the tensor
gradients at degenerate point positions.
\\
\\
Let $\vec{x}_0 = (x_0,y_0)$ be an isolated degenerate point. Assuming that the functions 
$T_{11}(\vec{x}) - T_{22}(\vec{x})$ and $T_{12}(\vec{x})$ are analytic, we can expand tensor
components in Taylor series around $\vec{x}_0$.
\\
$\vdots$
\\
$\vdots$
\\
$\left<\mbox{after some simplifications, we end up with the following - see pages
 113-114}\right>$
\begin{align}
\left\{\begin{matrix}
 \frac{T_{11} - T_{22}}{2} &\approx a(x - x_0) + b(y - y_0) + \cdots \\
 T_{12} &\approx c(x - x_0) + d(y - y_0) + \cdots
\end{matrix}\right.
\end{align}
where 
\begin{align*}
a &\equiv \frac{1}{2} \left.\frac{\partial (T_{11} - T_{22})}{\partial x}\right|_{x_0} &
b &\equiv \frac{1}{2} \left.\frac{\partial (T_{11} - T_{22})}{\partial y}\right|_{y_0} \\
c &\equiv \frac{1}{2} \left.\frac{\partial T_{12}}{\partial x}\right|_{x_0} &
d &\equiv \frac{1}{2} \left.\frac{\partial T_{12}}{\partial y}\right|_{y_0}
\end{align*}
An important quantity for characterizing degenerate points is
\begin{align}
\delta = ad - bc
\end{align}
which is invariant under rotation.$\left<\mbox{See pages 114 - 115 for proof.}\right>$
\begin{defn}[Simple and Multiple Degenerate Points] \label{definition}
Let $\vec{x}_0$ be an isolated degenerate point of a tensor field T $\in C^1(E)$, where $E$ is
an open subset of $\mathbf{R}^2$, and let $\delta = ad - bc$ be the corresponding third-order
invariant. Then, $\vec{x}_0$ is
\begin{itemize}
\item a simple degenerate point iff $\delta \neq 0$, or
\item a multiple degenerate point iff $\delta = 0$.
\end{itemize}
\end{defn}

\begin{thm}
The angle $\theta_k$ between the x-axis and the sepratrices $s_k$ are obtained by computing
the real roots $z_k$ of the cubic equation
\begin{align} \label{cubic}
dz^3 + (c + 2b)z^2 + (2a - d)z - c = 0,
\end{align}
by inverting the relation $z_k = tan \theta_k$, and by keeping only those angles that lie along 
the boundary of a hyperbolic sector.
\end{thm}
$\left<\mbox{for the whole theorem - see page 120}\right>$

\subsection{The Algorithm - see page 154}
The following algorithm handles simple degenerate points only :
\begin{enumerate}
\item locate degenerate points by searching for solutions to \eqref{degenerate_cond} in
every grid cell;
\item  classify each degenerate point as a trisector $(\delta < 0 )$ or a wedge point
$(\delta > 0 )$ by evaluating a,b,c,d and computing $\delta = ad - bc$;
\item select an eigenvector field;
\item solve \eqref{cubic} to find the directions of the three sepratrices $(s_1,s_2,s_3)$ at each
trisector point; likewise, extract sepratrices $(s_1,s_2)$ at wedge points where \eqref{cubic}
admits three real roots and extract the unique sepratrix at wedge points where \eqref{cubic}
has only one real root;
\item integrate hyperstreamlines along the sepratrices; terminate the trajectories wherever
they leave the domain, or impinge on the parabolic sector of a wedge point.
\end{enumerate}


\begin{thebibliography}{9}
\bibitem{1}
	T. Delmarcelle, 1994.
	\emph{The Visualization of Second-Order Tensor Fields}.
\end{thebibliography}

\end{document}