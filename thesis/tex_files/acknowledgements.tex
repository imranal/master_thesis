\documentclass[main.tex]{subfiles} 

\begin{document}

\section*{Abstract}      
Geodesics and hyperstreamlines are used to visualize second order tensors. 
We further look at a new way of visualizing second order tensor fields. By using 
the direction of geodesic curves in stead of eigenvectors, we make a different approach 
to so called integration methods. We extend the concept to include tensors which are not 
necessarily the metric.
\\\\
\hspace{-7mm}\textbf{Keywords}: Tensor Field Visualization, Hyperstreamlines, Geodesics       

\newpage\null

\section*{Acknowledgements}

Thesis supervisors :
\begin{itemize}
\item \O yvind Andreassen (FFI,UNIK),
\item Anders Helgeland (FFI,UNIK),
\item Kent-Andre Mardal (UiO).
\end{itemize}


This thesis uses extensively the programming language Python. 
For the sake of reproducibility, here we list the versions of the packages and sub-packages 
that are used throughout the thesis.
\begin{itemize}
\item Python - v.2.7.9 
\begin{itemize}
\item NumPy - v.1.8.2
\item SciPy - v.0.16.0
\item Matplotlib - v.1.4.2
\item SymPy - v.0.7.7.dev
\end{itemize}
\item IPython - v.2.3.0 
\end{itemize}
For SymPy, instead of using a stable release\footnote{https://github.com/sympy/sympy/releases}, 
we were required to use the latest developer version due to bugs that were present in the current
release\footnote{https://github.com/sympy/sympy/releases/tag/sympy-0.7.6.1}.
\\
\par
This may seem like a strange thing to do, but I would like to acknowledge the following.
A high-level language like Python has made this thesis a very fun and interesting task.
Instead of bogging down into details, Python has given me the leverage required to
focus on other aspects of the coding; where you are no longer required to exhibit strenuous efforts
to locate bugs and errors. Instead it allows the coder to focus efforts in quickly writing
down the mathematical or physical problem with relative ease, and visualising the results in
a similar quick manner. The best part
in all of this is that every package and sub-package used to achieve the results are free and
available online. Therefore, I am grateful to every person who has contributed in making Python into
the powerful mathematical tool that it has now become.
\\
\par
Last but not least, I am grateful to all my supervisors for their support and encouragement.
Frankly, I am befuddled that they managed to put up with my constant queries. In this regard,
to more than any, I am grateful to Professor \O yvind Andreassen.
\end{document}
